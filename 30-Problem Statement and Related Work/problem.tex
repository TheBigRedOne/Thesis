\chapter{Problem}
\label{c:problem}

This chapter formalises the producer mobility problem studied in this thesis. We first define the network and mobility model, then state the baseline behaviour that causes long service disruption under producer movement. Finally, we specify the evaluation goals and the metrics used throughout the remainder of the dissertation.

\section{System model}
\label{sec:problem:model}

We consider an NDN network consisting of a set of routers interconnected by point-to-point links. Consumers issue \emph{Interests} for named content and receive \emph{Data} packets on the reverse path created by PIT state. Routers forward Interests using the forwarding information base (FIB), and route computation is provided by a routing protocol (in our baseline, a link-state control plane).

\textbf{Producer mobility.}
A \emph{producer} serves a name prefix (e.g., a live stream prefix) and may move between attachment points (edge routers). A \emph{mobility event} occurs when the producer detaches from one access router and reattaches to another, potentially changing the forwarding path from consumers to the producer.

\textbf{Micro-mobility scope.}
The primary scope of this thesis is \emph{intra-domain} producer mobility (micro-mobility), where the producer remains within the same administrative domain and the naming scheme stays stable. Consumers are assumed to keep requesting the same name prefix across mobility events.

\textbf{Traffic pattern.}
We focus on continuous delivery workloads (e.g., live streaming or repetitive chunk retrieval) where consumers generate Interests at a steady pace and the producer responds with a steady stream of Data. This traffic pattern stresses continuity: even short disruptions can lead to visible degradation at the application layer.

\section{Baseline mobility behaviour in NDN}
\label{sec:problem:baseline}

Producer mobility is challenging in NDN because forwarding correctness depends on a consistent mapping between name prefixes and next hops. When the producer changes attachment point, the FIB entries that previously guided Interests towards the old location can become stale. During the interval before the control plane converges to the new best path, Interests may be forwarded toward the old attachment point and fail, resulting in missing Data and repeated retransmissions.

In a link-state control plane, convergence requires (i) neighbour/adjacency change detection, (ii) dissemination of updated routing advertisements, and (iii) recomputation and installation of new routes. These steps are periodic and timer-driven in practice, so convergence can be substantially slower than the underlying data plane forwarding timescale. Consequently, \emph{service disruption after producer movement is dominated by routing convergence delay rather than by packet transmission delay}.

\section{Problem statement}
\label{sec:problem:statement}

We study the following problem.

\begin{quote}
\textbf{Producer mobility continuity problem.}
Given a producer that changes its attachment point within a domain, how can the network maintain (or rapidly restore) Interest--Data exchange for ongoing consumers \emph{without waiting for global routing convergence}, while keeping overhead and safety risks bounded?
\end{quote}

This problem has three coupled aspects:

\begin{enumerate}
  \item \textbf{Continuity}: minimise the disruption perceived by applications (outage time, loss, and instability).
  \item \textbf{Efficiency}: limit additional traffic and state induced by any fast-recovery mechanism.
  \item \textbf{Safety}: ensure the recovery mechanism cannot be trivially abused to cause persistent flooding, routing corruption, or unbounded state growth.
\end{enumerate}

\section{Assumptions and design constraints}
\label{sec:problem:assumptions}

The evaluation and the mechanisms discussed later in this thesis adopt the following constraints, grounded in the behaviour of our prototype system:

\begin{itemize}
  \item \textbf{Point-to-point forwarding scope.} Flooding-based recovery, if used, is restricted to point-to-point faces and excludes local/control faces, to avoid accidental amplification through local-only channels. :contentReference[oaicite:0]{index=0}
  \item \textbf{Bounded flooding radius.} Any opportunistic broadcast-like forwarding must be constrained by hop limits. In our prototype, Interest flooding uses the native HopLimit field with a small default (e.g., 3) and decrements per hop. 
  \item \textbf{Temporary forwarding state.} Fast recovery may rely on short-lived forwarding hints/state (e.g., a temporary FIB) that expires quickly (seconds-scale) and is refreshed only when used, preventing indefinite persistence. Our prototype uses a temporary FIB (TFIB) with an idle expiry on the order of 5 \,s and a sliding refresh on use. 
  \item \textbf{Trigger conditions.} Recovery actions should be invoked only when the normal forwarding path is absent or clearly failing. In our prototype, Interest flooding is triggered when there is no usable next hop or upon receiving a \texttt{NO\_ROUTE} NACK, and then suppressed by short-term per-node de-duplication. 
\end{itemize}

These constraints intentionally bias the solution space toward mechanisms that are local, bounded, and fail-safe by construction.

\section{Evaluation goals}
\label{sec:problem:goals}

The remainder of the thesis evaluates mechanisms against the following goals:

\begin{enumerate}
  \item \textbf{Rapid restoration:} restore stable Interest--Data exchange shortly after a mobility event, ideally on the order of data-plane RTTs rather than control-plane convergence time.
  \item \textbf{High reliability during transition:} Keep the loss and retransmission pressure low during and after movement.
  \item \textbf{Low and bounded overhead:} any additional traffic (e.g., flooded Interests/Data) and state (e.g., temporary entries) should be bounded and short-lived.
  \item \textbf{Graceful fallback:} when fast recovery is not possible (e.g., misconfiguration or partial deployment), behaviour should degrade to the baseline rather than fail catastrophically.
\end{enumerate}

\section{Metrics}
\label{sec:problem:metrics}

We define a mobility event time as $t_m$. Let $t_s$ be the time when the consumer resumes receiving Data \emph{continuously} (i.e., sustained delivery rather than a single sporadic Data). Using these, we define:

\subsection{Continuity and performance}
\label{sec:problem:metrics:continuity}

\textbf{Recovery time (duration of the outage).}
\begin{equation}
  T_{\mathrm{rec}} = t_s - t_m.
\end{equation}

\textbf{Interest satisfaction ratio.}
Over an interval $[t_1,t_2]$, let $N_I$ be the number of Interests emitted (including retransmissions) and $N_D$ the number of Data received by the consumer:
\begin{equation}
  R_{\mathrm{sat}} = \frac{N_D}{N_I}.
\end{equation}
This captures both the disruption and the retransmission pressure.

\textbf{Good performance.}
We measure application-visible throughput as payload bytes of Data delivered per second at the consumer. For streaming-like workloads, we additionally report time-series goodput to show stability after recovery.

\subsection{Overhead}
\label{sec:problem:metrics:overhead}

\textbf{Flooding overhead.}
We quantify overhead as the excess number of packets/bytes forwarded relative to baseline, separated by packet type:
\begin{equation}
  O_X = \frac{\mathrm{bytes\_fwd}_X - \mathrm{bytes\_fwd}^{\mathrm{base}}_X}{\mathrm{bytes\_fwd}^{\mathrm{base}}_X}, \quad X \in \{\mathrm{Interest},\mathrm{Data}\}.
\end{equation}
In the prototype, flooding is explicitly bounded by hop limits and de-duplication; thus overhead should manifest as short bursts around mobility events rather than sustained increases. 

\textbf{Control-plane overhead (when applicable).}
When a mechanism interacts with routing (e.g., short-lived routing hints), we report the number of control messages and the duration for which the temporary routing state remains installed.

\subsection{State and safety indicators}
\label{sec:problem:metrics:safety}

\textbf{Temporary state footprint.}
We measure the number of temporary forwarding entries (e.g., TFIB size over time) and their lifetimes. A desirable property is fast decay after the mobility transient; in our prototype, TFIB is explicitly short-lived (second-scale) and refreshed only by use. 

\textbf{Duplication effectiveness.}
We report the observed rate of duplicates suppressed by per-packet identifiers/caches (for both Data and Interest flooding), since duplication directly drives overhead and potential instability. 

\section{Summary}
\label{sec:problem:summary}

Producer mobility in NDN creates a mismatch between stable naming at the application layer and time-varying attachment points at the network edge. Under a routing-driven baseline, the resulting disruption is dominated by control-plane convergence. This thesis therefore seeks mechanisms that provide rapid, bounded, and safe continuity during the convergence window, evaluated by recovery time, reliability, overhead, and temporary state behaviour.
